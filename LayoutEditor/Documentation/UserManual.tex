\documentclass[11pt,a4paper]{report}

% Packages
\usepackage[utf8]{inputenc}
\usepackage[T1]{fontenc}
\usepackage{lmodern}
\usepackage[margin=1in,headheight=14pt]{geometry}
\usepackage{graphicx}
\usepackage{xcolor}
\usepackage{listings}
\usepackage{booktabs}
\usepackage{longtable}
\usepackage{array}
\usepackage{multirow}
\usepackage{hyperref}
\usepackage{fancyhdr}
\usepackage{titlesec}
\usepackage{enumitem}
\usepackage{tcolorbox}
\usepackage{menukeys}
\usepackage{fontawesome5}

% Colors
\definecolor{primaryblue}{RGB}{74, 144, 217}
\definecolor{successgreen}{RGB}{46, 204, 113}
\definecolor{warningorange}{RGB}{243, 156, 18}
\definecolor{errorred}{RGB}{231, 76, 60}
\definecolor{codebg}{RGB}{248, 249, 250}
\definecolor{darkgray}{RGB}{73, 80, 87}

% Hyperref setup
\hypersetup{
    colorlinks=true,
    linkcolor=primaryblue,
    urlcolor=primaryblue,
    citecolor=primaryblue,
    pdftitle={Simulation Layout Editor - User Manual},
    pdfauthor={Simulation Team},
    pdfsubject={User Manual}
}

% Header/Footer
\pagestyle{fancy}
\fancyhf{}
\fancyhead[L]{\leftmark}
\fancyhead[R]{Simulation Layout Editor}
\fancyfoot[C]{\thepage}
\renewcommand{\headrulewidth}{0.4pt}
\renewcommand{\footrulewidth}{0.4pt}

% Chapter formatting
\titleformat{\chapter}[display]
{\normalfont\huge\bfseries\color{primaryblue}}{\chaptertitlename\ \thechapter}{20pt}{\Huge}

% Code listing style
\lstdefinelanguage{json}{
    basicstyle=\ttfamily\small,
    numbers=left,
    numberstyle=\tiny\color{gray},
    stepnumber=1,
    numbersep=8pt,
    showstringspaces=false,
    breaklines=true,
    literate=
     *{0}{{{\color{primaryblue}0}}}{1}
      {1}{{{\color{primaryblue}1}}}{1}
      {2}{{{\color{primaryblue}2}}}{1}
      {3}{{{\color{primaryblue}3}}}{1}
      {4}{{{\color{primaryblue}4}}}{1}
      {5}{{{\color{primaryblue}5}}}{1}
      {6}{{{\color{primaryblue}6}}}{1}
      {7}{{{\color{primaryblue}7}}}{1}
      {8}{{{\color{primaryblue}8}}}{1}
      {9}{{{\color{primaryblue}9}}}{1}
      {:}{{{\color{darkgray}{:}}}}{1}
      {,}{{{\color{darkgray}{,}}}}{1}
      {\{}{{{\color{darkgray}{\{}}}}{1}
      {\}}{{{\color{darkgray}{\}}}}}{1}
      {[}{{{\color{darkgray}{[}}}}{1}
      {]}{{{\color{darkgray}{]}}}}{1},
    morestring=[b]",
    stringstyle=\color{successgreen},
}

\lstset{
    backgroundcolor=\color{codebg},
    basicstyle=\ttfamily\small,
    breaklines=true,
    frame=single,
    rulecolor=\color{gray!30},
    numbers=left,
    numberstyle=\tiny\color{gray},
    keywordstyle=\color{primaryblue},
    stringstyle=\color{successgreen},
    commentstyle=\color{gray},
    tabsize=2
}

% Custom boxes
\newtcolorbox{notebox}{
    colback=primaryblue!10,
    colframe=primaryblue,
    title=\faInfoCircle\ Note,
    fonttitle=\bfseries
}

\newtcolorbox{tipbox}{
    colback=successgreen!10,
    colframe=successgreen,
    title=\faLightbulb\ Tip,
    fonttitle=\bfseries
}

\newtcolorbox{warningbox}{
    colback=warningorange!10,
    colframe=warningorange,
    title=\faExclamationTriangle\ Warning,
    fonttitle=\bfseries
}

\newtcolorbox{importantbox}{
    colback=errorred!10,
    colframe=errorred,
    title=\faExclamationCircle\ Important,
    fonttitle=\bfseries
}

% Custom commands
\newcommand{\menupath}[1]{\texttt{\color{darkgray}#1}}
\newcommand{\keystroke}[1]{\texttt{\fcolorbox{gray!50}{gray!10}{#1}}}
\newcommand{\button}[1]{\texttt{[#1]}}

% Document info
\title{
    \vspace{-2cm}
    \includegraphics[width=0.3\textwidth]{placeholder}\\[1cm]
    {\Huge\bfseries\color{primaryblue}Simulation Layout Editor}\\[0.5cm]
    {\LARGE User Manual}\\[0.3cm]
    {\large Version 2.0}
}
\author{Simulation Team}
\date{\today}

\begin{document}

% Title page
\begin{titlepage}
    \centering
    \vspace*{2cm}
    
    {\Huge\bfseries\color{primaryblue}Simulation Layout Editor}\\[1cm]
    {\LARGE User Manual}\\[0.5cm]
    {\large Version 2.0}
    
    \vspace{2cm}
    
    \begin{tcolorbox}[colback=white,colframe=primaryblue,width=0.8\textwidth]
        \centering
        \large
        A visual editor for creating discrete event simulation layouts with support for manufacturing systems, AGV routing, and process modeling.
    \end{tcolorbox}
    
    \vfill
    
    {\large\today}
    
\end{titlepage}

% Table of contents
\tableofcontents
\newpage

% ============================================================================
\chapter{Introduction}
% ============================================================================

\section{Overview}

The \textbf{Simulation Layout Editor} is a visual design tool for creating discrete event simulation layouts. It provides an intuitive interface for designing manufacturing floors, production lines, warehouse systems, and other process-oriented environments.

\subsection{Key Features}

\begin{itemize}[leftmargin=*]
    \item \textbf{Visual Layout Design} -- Drag-and-drop placement of simulation elements
    \item \textbf{Multiple Node Types} -- Sources, sinks, machines, buffers, workstations, and more
    \item \textbf{Path Routing} -- Single and double lane paths with direct or Manhattan routing
    \item \textbf{Simulation Parameters} -- Configure process times, capacities, and distributions
    \item \textbf{Validation} -- Automatic detection of layout issues
    \item \textbf{JSON Export} -- Standard format for simulation engine consumption
    \item \textbf{Icon Library} -- 40+ industrial vector icons
\end{itemize}

\subsection{System Requirements}

\begin{table}[h]
\centering
\begin{tabular}{ll}
\toprule
\textbf{Component} & \textbf{Requirement} \\
\midrule
Operating System & Windows 10/11 (64-bit) \\
Runtime & .NET 8.0 Desktop Runtime \\
Memory & 4 GB RAM minimum \\
Display & 1280 $\times$ 720 minimum resolution \\
\bottomrule
\end{tabular}
\caption{System requirements}
\end{table}

\section{Installation}

\subsection{Prerequisites}

Ensure the .NET 8.0 Desktop Runtime is installed. Download from:
\begin{center}
\url{https://dotnet.microsoft.com/download/dotnet/8.0}
\end{center}

\subsection{Running the Application}

\begin{enumerate}
    \item Extract the \texttt{LayoutEditor.zip} archive
    \item Navigate to the extracted folder
    \item Run \texttt{dotnet build} to compile
    \item Run \texttt{dotnet run} to start the application
\end{enumerate}

Alternatively, if you have a compiled executable:
\begin{enumerate}
    \item Double-click \texttt{LayoutEditor.exe}
\end{enumerate}

% ============================================================================
\chapter{User Interface}
% ============================================================================

\section{Main Window Layout}

The application window is divided into several functional areas:

\begin{figure}[h]
\centering
\begin{tcolorbox}[colback=white,colframe=gray,width=\textwidth]
\begin{verbatim}
+------------------------------------------------------------------+
|  Menu Bar                                                         |
+------------------------------------------------------------------+
|  Toolbar (File, Edit, Tools, Zoom, Validation)                   |
+------------+-------------------------------------+----------------+
|            |                                     |                |
|  Toolbox   |                                     |   Properties   |
|  & Layers  |         Canvas Area                 |     Panel      |
|            |                                     |                |
|  - Nodes   |     [Grid and Layout Elements]      |  - Basic Info  |
|  - Layers  |                                     |  - Position    |
|  - Elements|                          [Minimap]  |  - Simulation  |
|            |                                     |                |
+------------+-------------------------------------+----------------+
|  Status Bar (Mode, Coordinates, Units, Counts)                   |
+------------------------------------------------------------------+
\end{verbatim}
\end{tcolorbox}
\caption{Main window layout}
\end{figure}

\section{Menu Bar}

\subsection{File Menu}

\begin{longtable}{p{4cm}p{3cm}p{7cm}}
\toprule
\textbf{Command} & \textbf{Shortcut} & \textbf{Description} \\
\midrule
\endhead
New & \keystroke{Ctrl+N} & Create a new empty layout \\
Open & \keystroke{Ctrl+O} & Open an existing layout file \\
Save & \keystroke{Ctrl+S} & Save current layout \\
Save As & -- & Save with a new filename \\
Import Background & -- & Load a floor plan image \\
Import DXF & -- & Import CAD drawing (future) \\
Export for Simulation & -- & Export simulation-ready JSON \\
Export as Image & -- & Save canvas as image (future) \\
Exit & -- & Close application \\
\bottomrule
\caption{File menu commands}
\end{longtable}

\subsection{Edit Menu}

\begin{longtable}{p{4cm}p{3cm}p{7cm}}
\toprule
\textbf{Command} & \textbf{Shortcut} & \textbf{Description} \\
\midrule
\endhead
Undo & \keystroke{Ctrl+Z} & Undo last action (50 levels) \\
Redo & \keystroke{Ctrl+Y} & Redo undone action \\
Cut & \keystroke{Ctrl+X} & Cut selected elements \\
Copy & \keystroke{Ctrl+C} & Copy selected elements \\
Paste & \keystroke{Ctrl+V} & Paste elements \\
Delete & \keystroke{Delete} & Delete selected elements \\
Select All & \keystroke{Ctrl+A} & Select all nodes \\
Group & \keystroke{Ctrl+G} & Group selected nodes \\
Ungroup & \keystroke{Ctrl+Shift+G} & Ungroup selection \\
\bottomrule
\caption{Edit menu commands}
\end{longtable}

\subsection{View Menu}

\begin{longtable}{p{4cm}p{3cm}p{7cm}}
\toprule
\textbf{Command} & \textbf{Shortcut} & \textbf{Description} \\
\midrule
\endhead
Zoom In & \keystroke{Ctrl++} & Increase zoom level \\
Zoom Out & \keystroke{Ctrl+-} & Decrease zoom level \\
Zoom to Fit & \keystroke{Ctrl+0} & Fit all content in view \\
Reset Zoom & -- & Return to 100\% zoom \\
Show Grid & -- & Toggle grid visibility \\
Show Rulers & -- & Toggle ruler visibility \\
Show Minimap & -- & Toggle minimap overlay \\
Show Labels & -- & Toggle node labels \\
Snap to Grid & -- & Toggle grid snapping \\
Show Alignment Guides & -- & Toggle alignment guides \\
\bottomrule
\caption{View menu commands}
\end{longtable}

\section{Toolbar}

The toolbar provides quick access to common operations:

\begin{table}[h]
\centering
\begin{tabular}{ccp{8cm}}
\toprule
\textbf{Icon} & \textbf{Name} & \textbf{Function} \\
\midrule
\faFile & New & Create new layout \\
\faFolderOpen & Open & Open existing file \\
\faSave & Save & Save current file \\
\faUndo & Undo & Undo last action \\
\faRedo & Redo & Redo action \\
\faMousePointer & Select & Selection mode (default) \\
\faHandPaper & Pan & Pan/scroll mode \\
\faArrowRight & Path & Draw path mode \\
\faSquare & Zone & Draw zone mode \\
\faMinus & Zoom Out & Decrease zoom \\
\faPlus & Zoom In & Increase zoom \\
\faCheck & Validate & Run validation \\
\bottomrule
\end{tabular}
\caption{Toolbar buttons}
\end{table}

\section{Left Panel: Toolbox \& Layers}

\subsection{Node Toolbox}

The toolbox contains all available node types organized by category:

\subsubsection{Sources \& Sinks}
\begin{itemize}
    \item \textbf{Source} -- Entry point for entities into the system
    \item \textbf{Sink/Exit} -- Exit point where entities leave the system
\end{itemize}

\subsubsection{Processing}
\begin{itemize}
    \item \textbf{Machine} -- Processing station with configurable process time
    \item \textbf{Workstation} -- Manual or semi-automated work area
    \item \textbf{Inspection} -- Quality control checkpoint
\end{itemize}

\subsubsection{Buffers \& Storage}
\begin{itemize}
    \item \textbf{Buffer (FIFO)} -- First-in-first-out queue
    \item \textbf{Storage/Rack} -- General storage area
\end{itemize}

\subsubsection{Transport}
\begin{itemize}
    \item \textbf{Conveyor} -- Material transport line
    \item \textbf{Junction} -- Path intersection point
    \item \textbf{AGV Station} -- Automated guided vehicle stop
\end{itemize}

\subsection{Layers Panel}

Control visibility of different element types:

\begin{itemize}
    \item \textbf{Background} -- Floor plan or reference image
    \item \textbf{Corridors} -- AGV/transport corridors
    \item \textbf{Zones} -- Restricted or special areas
    \item \textbf{Paths} -- Connection lines between nodes
    \item \textbf{Nodes} -- All simulation elements
    \item \textbf{Labels} -- Node names and identifiers
\end{itemize}

\subsection{Elements List}

Displays all nodes and paths in the current layout. Click an item to select it on the canvas.

\section{Right Panel: Properties}

The properties panel shows details of the selected element.

\subsection{Node Properties}

\subsubsection{Basic Information}
\begin{itemize}
    \item \textbf{ID} -- Unique identifier (read-only)
    \item \textbf{Type} -- Node type (source, machine, etc.)
    \item \textbf{Name} -- Descriptive name
    \item \textbf{Label} -- Short label shown on canvas
\end{itemize}

\subsubsection{Position \& Size}
\begin{itemize}
    \item \textbf{X, Y} -- Canvas coordinates
    \item \textbf{Width, Height} -- Node dimensions
    \item \textbf{Rotation} -- Rotation angle in degrees
\end{itemize}

\subsubsection{Visual}
\begin{itemize}
    \item \textbf{Icon} -- Visual representation
    \item \textbf{Color} -- Node color (hex format)
    \item \textbf{Label Position} -- Bottom, Top, Left, Right, Center
\end{itemize}

\subsubsection{Simulation Parameters}

Parameters vary by node type. See Chapter~\ref{ch:simulation} for details.

\subsection{Path Properties}

\begin{itemize}
    \item \textbf{From/To} -- Connected nodes
    \item \textbf{Path Type} -- Single or Double lane
    \item \textbf{Routing Mode} -- Direct, Manhattan, or Corridor
    \item \textbf{Transport Type} -- Conveyor, AGV, Manual, Crane
    \item \textbf{Speed} -- Transport speed
    \item \textbf{Capacity} -- Maximum entities on path
\end{itemize}

\section{Canvas Area}

The central canvas is where you design your layout.

\subsection{Navigation}

\begin{table}[h]
\centering
\begin{tabular}{lp{8cm}}
\toprule
\textbf{Action} & \textbf{How To} \\
\midrule
Pan & Hold \keystroke{Space} and drag, or use Pan tool \\
Zoom In/Out & \keystroke{Ctrl} + Mouse Wheel \\
Zoom to Fit & \keystroke{Ctrl+0} or toolbar button \\
\bottomrule
\end{tabular}
\caption{Canvas navigation}
\end{table}

\subsection{Grid and Snapping}

\begin{itemize}
    \item Grid lines help align elements
    \item When ``Snap to Grid'' is enabled, elements align to grid intersections
    \item Default grid size is 20 pixels
\end{itemize}

\subsection{Minimap}

The minimap in the bottom-right corner shows an overview of the entire layout. The blue rectangle indicates the current viewport.

\section{Status Bar}

The status bar displays:

\begin{itemize}
    \item \textbf{Status} -- Current operation or message
    \item \textbf{Mode} -- Active tool (Select, Pan, Path, etc.)
    \item \textbf{Coordinates} -- Mouse position in current units
    \item \textbf{Units} -- Current measurement unit (meters, feet, pixels)
    \item \textbf{Counts} -- Number of nodes and paths
    \item \textbf{File} -- Current filename and save status
\end{itemize}

% ============================================================================
\chapter{Working with Nodes}
% ============================================================================

\section{Adding Nodes}

There are several ways to add nodes to the canvas:

\subsection{Method 1: Toolbox Button}

\begin{enumerate}
    \item Click a node type button in the left toolbox
    \item The node appears at the center of the visible canvas
    \item Drag to reposition as needed
\end{enumerate}

\subsection{Method 2: Right-Click Menu}

\begin{enumerate}
    \item Right-click on the canvas at the desired location
    \item Select \menupath{Add Node Here} from the context menu
    \item Choose the node type from the submenu
\end{enumerate}

\begin{tipbox}
Using the right-click method places the node exactly where you click, saving time on positioning.
\end{tipbox}

\section{Selecting Nodes}

\begin{table}[h]
\centering
\begin{tabular}{lp{8cm}}
\toprule
\textbf{Action} & \textbf{Result} \\
\midrule
Click node & Select single node (deselects others) \\
\keystroke{Ctrl} + Click & Toggle node in selection \\
\keystroke{Shift} + Click & Add node to selection \\
\keystroke{Ctrl+A} & Select all nodes \\
Click empty space & Deselect all \\
\bottomrule
\end{tabular}
\caption{Selection methods}
\end{table}

Selected nodes display a blue glow effect and thicker borders.

\section{Moving Nodes}

\begin{enumerate}
    \item Select one or more nodes
    \item Click and drag any selected node
    \item All selected nodes move together
    \item Release to place
\end{enumerate}

\begin{notebox}
When ``Snap to Grid'' is enabled, nodes automatically align to grid points while dragging.
\end{notebox}

\section{Editing Node Properties}

\begin{enumerate}
    \item Select a single node
    \item View properties in the right panel
    \item Edit values directly in the text fields
    \item Changes apply immediately
\end{enumerate}

\section{Duplicating Nodes}

\begin{enumerate}
    \item Select one or more nodes
    \item Right-click and select \menupath{Duplicate}
    \item Copies appear offset from originals
    \item New nodes are automatically selected
\end{enumerate}

\section{Deleting Nodes}

\begin{enumerate}
    \item Select nodes to delete
    \item Press \keystroke{Delete} or use \menupath{Edit > Delete}
\end{enumerate}

\begin{warningbox}
Deleting a node also deletes all paths connected to it.
\end{warningbox}

\section{Aligning Multiple Nodes}

When multiple nodes are selected, alignment tools appear in the properties panel:

\begin{itemize}
    \item \textbf{Align Left} -- Align to leftmost node
    \item \textbf{Align Top} -- Align to topmost node
    \item \textbf{Distribute Horizontally} -- Equal horizontal spacing
    \item \textbf{Distribute Vertically} -- Equal vertical spacing
\end{itemize}

% ============================================================================
\chapter{Working with Paths}
% ============================================================================

\section{Path Concepts}

Paths represent material flow or entity movement between nodes. Each path has:

\begin{itemize}
    \item A \textbf{source node} (where entities come from)
    \item A \textbf{destination node} (where entities go)
    \item Optional \textbf{waypoints} for custom routing
    \item \textbf{Visual properties} (color, thickness, style)
    \item \textbf{Simulation properties} (speed, capacity, transport type)
\end{itemize}

\section{Path Types}

\subsection{Single Lane}

Standard unidirectional path. Drawn as a single line with an arrow indicating direction.

\subsection{Double Lane}

Higher-capacity path rendered as two parallel lines. Useful for:
\begin{itemize}
    \item High-throughput connections
    \item Bidirectional transport (visual indication)
    \item Multi-lane conveyors
\end{itemize}

\section{Routing Modes}

\subsection{Direct Routing}

Straight line from source to destination. Best for:
\begin{itemize}
    \item Short connections
    \item Adjacent nodes
    \item Simple layouts
\end{itemize}

\subsection{Manhattan Routing}

Paths use only horizontal and vertical segments (90° turns). Best for:
\begin{itemize}
    \item Grid-based layouts
    \item Conveyor systems
    \item Clean, organized appearance
\end{itemize}

\section{Creating Paths}

\subsection{Method 1: Path Tool}

\begin{enumerate}
    \item Select the Path tool from the toolbar (or press \keystroke{P})
    \item Click the source node
    \item (Optional) \keystroke{Ctrl} + Click to add waypoints
    \item Click the destination node
\end{enumerate}

\subsection{Method 2: Right-Click Menu}

\begin{enumerate}
    \item Right-click on the source node
    \item Select \menupath{Start Path From Here}
    \item Right-click on the destination node
    \item Select \menupath{End Path Here}
\end{enumerate}

\subsection{Setting Path Options Before Drawing}

Before creating a path, set options in the toolbar:
\begin{itemize}
    \item \textbf{Path Type} dropdown -- Single or Double
    \item \textbf{Manhattan} checkbox -- Enable orthogonal routing
\end{itemize}

\section{Adding Waypoints}

Waypoints allow custom path routing:

\begin{enumerate}
    \item Start path drawing (click source node)
    \item Hold \keystroke{Ctrl} and click to add waypoint
    \item Add additional waypoints as needed
    \item Click destination node to complete
\end{enumerate}

\begin{tipbox}
Waypoints are useful for routing paths around obstacles or through specific corridors.
\end{tipbox}

\section{Editing Paths}

\begin{enumerate}
    \item Click on a path line to select it (or use Elements list)
    \item Edit properties in the right panel
    \item Changes to From/To reconnect the path to different nodes
\end{enumerate}

\section{Deleting Paths}

\begin{enumerate}
    \item Select the path
    \item Press \keystroke{Delete} or click \button{Delete Path} in properties panel
\end{enumerate}

% ============================================================================
\chapter{Simulation Parameters}
\label{ch:simulation}
% ============================================================================

\section{Overview}

Each node type has specific simulation parameters that control behavior during simulation execution.

\section{Statistical Distributions}

Many parameters use statistical distributions to model variability:

\begin{table}[h]
\centering
\begin{tabular}{lp{5cm}l}
\toprule
\textbf{Distribution} & \textbf{Parameters} & \textbf{Use Case} \\
\midrule
Constant & Value & Fixed times \\
Exponential & Mean ($\mu$) & Random arrivals \\
Normal & Mean ($\mu$), Std Dev ($\sigma$) & Natural variation \\
Uniform & Min, Max & Equal probability range \\
Triangular & Min, Mode, Max & Estimated ranges \\
Weibull & Shape (k), Scale ($\lambda$) & Failure modeling \\
\bottomrule
\end{tabular}
\caption{Available statistical distributions}
\end{table}

\section{Node-Specific Parameters}

\subsection{Source Nodes}

\begin{table}[h]
\centering
\begin{tabular}{lp{8cm}}
\toprule
\textbf{Parameter} & \textbf{Description} \\
\midrule
Interarrival Time & Time between entity arrivals (distribution) \\
Entity Type & Type/class of generated entities \\
Batch Size & Number of entities per arrival \\
Max Arrivals & Limit on total arrivals (optional) \\
\bottomrule
\end{tabular}
\caption{Source node parameters}
\end{table}

\begin{notebox}
For an arrival rate of $\lambda$ entities per hour, set the interarrival time to an exponential distribution with mean $\mu = 60/\lambda$ minutes.
\end{notebox}

\subsection{Machine \& Workstation Nodes}

\begin{table}[h]
\centering
\begin{tabular}{lp{8cm}}
\toprule
\textbf{Parameter} & \textbf{Description} \\
\midrule
Servers & Number of parallel processing units \\
Capacity & Maximum entities that can be processed simultaneously \\
Process Time & Time to process one entity (distribution) \\
Setup Time & Time between different entity types (optional) \\
MTBF & Mean Time Between Failures (hours) \\
MTTR & Mean Time To Repair (minutes) \\
\bottomrule
\end{tabular}
\caption{Machine/Workstation parameters}
\end{table}

\subsection{Buffer \& Storage Nodes}

\begin{table}[h]
\centering
\begin{tabular}{lp{8cm}}
\toprule
\textbf{Parameter} & \textbf{Description} \\
\midrule
Capacity & Maximum number of entities \\
Initial Level & Entities present at simulation start \\
Queue Discipline & FIFO, LIFO, or Priority \\
Blocking Mode & Behavior when full \\
\bottomrule
\end{tabular}
\caption{Buffer/Storage parameters}
\end{table}

\subsection{Inspection Nodes}

\begin{table}[h]
\centering
\begin{tabular}{lp{8cm}}
\toprule
\textbf{Parameter} & \textbf{Description} \\
\midrule
Servers & Number of inspection stations \\
Process Time & Inspection duration (distribution) \\
\bottomrule
\end{tabular}
\caption{Inspection node parameters}
\end{table}

\subsection{Sink Nodes}

\begin{table}[h]
\centering
\begin{tabular}{lp{8cm}}
\toprule
\textbf{Parameter} & \textbf{Description} \\
\midrule
Collect Statistics & Enable/disable output statistics collection \\
\bottomrule
\end{tabular}
\caption{Sink node parameters}
\end{table}

\section{Path Parameters}

\begin{table}[h]
\centering
\begin{tabular}{lp{8cm}}
\toprule
\textbf{Parameter} & \textbf{Description} \\
\midrule
Transport Type & Conveyor, AGV, Manual, or Crane \\
Speed & Transport velocity (units per time) \\
Capacity & Maximum entities on path simultaneously \\
Distance & Path length (auto-calculated or manual) \\
Lanes & Number of parallel lanes \\
Bidirectional & Allow two-way travel \\
\bottomrule
\end{tabular}
\caption{Path simulation parameters}
\end{table}

% ============================================================================
\chapter{Validation}
% ============================================================================

\section{Overview}

The validation system checks your layout for potential issues before simulation. Run validation via:
\begin{itemize}
    \item \menupath{Validate > Validate Layout}
    \item Toolbar validate button
    \item Keyboard shortcut \keystroke{F5}
\end{itemize}

\section{Validation Rules}

\subsection{Error-Level Issues}

Errors indicate problems that will likely cause simulation failures:

\begin{table}[h]
\centering
\begin{tabular}{lp{7cm}}
\toprule
\textbf{Code} & \textbf{Description} \\
\midrule
ZERO\_SERVERS & Machine has 0 servers configured \\
ZERO\_CAPACITY & Buffer has 0 capacity \\
MISSING\_PROCESS\_TIME & Machine lacks process time distribution \\
MISSING\_INTERARRIVAL & Source lacks interarrival time \\
INVALID\_DISTRIBUTION & Distribution parameters are invalid \\
UNREACHABLE\_SINK & No path exists from any source to sink \\
\bottomrule
\end{tabular}
\caption{Error-level validation issues}
\end{table}

\subsection{Warning-Level Issues}

Warnings indicate potential problems or unusual configurations:

\begin{table}[h]
\centering
\begin{tabular}{lp{7cm}}
\toprule
\textbf{Code} & \textbf{Description} \\
\midrule
DISCONNECTED\_NODE & Node has no incoming or outgoing paths \\
ORPHAN\_SOURCE & Source has no outgoing paths \\
ORPHAN\_SINK & Sink has no incoming paths \\
NO\_SOURCES & Layout has no source nodes \\
NO\_SINKS & Layout has no sink nodes \\
OVERLAPPING\_NODES & Two nodes visually overlap \\
PATH\_CROSSES\_RESTRICTED & Path goes through restricted zone \\
\bottomrule
\end{tabular}
\caption{Warning-level validation issues}
\end{table}

\section{Interpreting Results}

The validation status appears in the toolbar:

\begin{itemize}
    \item \textcolor{successgreen}{\faCheck\ Valid} -- No issues found
    \item \textcolor{warningorange}{\faExclamationTriangle\ X errors, Y warnings} -- Issues detected
\end{itemize}

Click the validation button to see a detailed report of all issues.

% ============================================================================
\chapter{File Operations}
% ============================================================================

\section{File Format}

Layouts are saved as JSON files with the \texttt{.json} extension. The format includes:

\begin{itemize}
    \item Metadata (name, author, units)
    \item Canvas settings
    \item Templates
    \item All nodes with visual and simulation properties
    \item All paths with routing and transport settings
    \item Zones and groups
    \item Display preferences
\end{itemize}

\section{Saving Layouts}

\subsection{Save}
\keystroke{Ctrl+S} saves to the current file. If no file exists, prompts for filename.

\subsection{Save As}
Creates a new file with a different name, leaving the original unchanged.

\section{Opening Layouts}

\keystroke{Ctrl+O} opens a file browser to select an existing layout file.

\begin{warningbox}
Opening a file replaces the current layout. Save your work first if needed.
\end{warningbox}

\section{Exporting for Simulation}

\menupath{File > Export for Simulation} creates a streamlined JSON file containing only simulation-relevant data (removes visual properties for smaller file size).

\section{Importing Background Images}

\begin{enumerate}
    \item \menupath{File > Import Background}
    \item Select an image file (PNG, JPG, BMP)
    \item Adjust opacity in canvas settings
    \item Use as reference for node placement
\end{enumerate}

\begin{tipbox}
Set background opacity to 20-30\% for best visibility while still seeing your layout elements.
\end{tipbox}

% ============================================================================
\chapter{Keyboard Shortcuts}
% ============================================================================

\section{Quick Reference}

\begin{longtable}{lp{8cm}}
\toprule
\textbf{Shortcut} & \textbf{Action} \\
\midrule
\endhead
\multicolumn{2}{l}{\textbf{File Operations}} \\
\keystroke{Ctrl+N} & New layout \\
\keystroke{Ctrl+O} & Open layout \\
\keystroke{Ctrl+S} & Save layout \\
\midrule
\multicolumn{2}{l}{\textbf{Edit Operations}} \\
\keystroke{Ctrl+Z} & Undo \\
\keystroke{Ctrl+Y} & Redo \\
\keystroke{Ctrl+X} & Cut \\
\keystroke{Ctrl+C} & Copy \\
\keystroke{Ctrl+V} & Paste \\
\keystroke{Delete} & Delete selected \\
\keystroke{Ctrl+A} & Select all \\
\keystroke{Ctrl+G} & Group selected \\
\keystroke{Ctrl+Shift+G} & Ungroup \\
\midrule
\multicolumn{2}{l}{\textbf{View Operations}} \\
\keystroke{Ctrl+Scroll} & Zoom in/out \\
\keystroke{Ctrl+0} & Zoom to fit \\
\keystroke{Space+Drag} & Pan canvas \\
\midrule
\multicolumn{2}{l}{\textbf{Tools}} \\
\keystroke{Escape} & Cancel current operation \\
\keystroke{F5} & Validate layout \\
\midrule
\multicolumn{2}{l}{\textbf{Path Drawing}} \\
\keystroke{Ctrl+Click} & Add waypoint (in path mode) \\
\midrule
\multicolumn{2}{l}{\textbf{Selection}} \\
\keystroke{Ctrl+Click} & Toggle selection \\
\keystroke{Shift+Click} & Add to selection \\
\bottomrule
\caption{Keyboard shortcuts}
\end{longtable}

% ============================================================================
\chapter{Best Practices}
% ============================================================================

\section{Layout Design Tips}

\begin{enumerate}
    \item \textbf{Start with sources and sinks} -- Define entry and exit points first
    \item \textbf{Work left-to-right} -- Typical flow direction aids readability
    \item \textbf{Use consistent spacing} -- Enable grid snapping for alignment
    \item \textbf{Group related elements} -- Use the Group feature for logical sections
    \item \textbf{Label clearly} -- Use descriptive names for all elements
    \item \textbf{Validate often} -- Check for issues during design, not just at the end
\end{enumerate}

\section{Naming Conventions}

\begin{table}[h]
\centering
\begin{tabular}{ll}
\toprule
\textbf{Element} & \textbf{Suggested Format} \\
\midrule
Sources & ``Raw Material Input'', ``Part Arrival'' \\
Machines & ``CNC Mill 1'', ``Drill Press A'' \\
Buffers & ``WIP Buffer'', ``Input Queue'' \\
Workstations & ``Assembly Station 1'', ``Packing'' \\
Sinks & ``Shipping'', ``Finished Goods Exit'' \\
\bottomrule
\end{tabular}
\caption{Naming conventions}
\end{table}

\section{Performance Considerations}

\begin{itemize}
    \item Large layouts (100+ nodes) may experience slower rendering
    \item Disable minimap for very large layouts if needed
    \item Use layers to hide elements you're not currently editing
    \item Save frequently during extended editing sessions
\end{itemize}

% ============================================================================
\chapter{Troubleshooting}
% ============================================================================

\section{Common Issues}

\subsection{Application Won't Start}

\begin{itemize}
    \item Ensure .NET 8.0 Desktop Runtime is installed
    \item Check Windows Event Viewer for error messages
    \item Try running from command line to see error output
\end{itemize}

\subsection{Cannot Save File}

\begin{itemize}
    \item Verify write permissions for the target folder
    \item Check if file is open in another application
    \item Ensure sufficient disk space
\end{itemize}

\subsection{Nodes Not Visible}

\begin{itemize}
    \item Check if ``Nodes'' layer is enabled in Layers panel
    \item Zoom out to see if nodes are outside visible area
    \item Use ``Zoom to Fit'' (\keystroke{Ctrl+0})
\end{itemize}

\subsection{Paths Not Connecting Properly}

\begin{itemize}
    \item Ensure you're clicking on the node, not empty space
    \item Check that both nodes exist and aren't deleted
    \item Verify ``Paths'' layer is visible
\end{itemize}

\section{Getting Help}

For additional support:
\begin{itemize}
    \item Check \menupath{Help > User Guide}
    \item View \menupath{Help > Keyboard Shortcuts}
    \item Contact your simulation team administrator
\end{itemize}

% ============================================================================
\appendix
\chapter{JSON Schema Reference}
% ============================================================================

\section{Root Structure}

\begin{lstlisting}[language=json,caption={Root JSON structure}]
{
  "version": "2.0",
  "metadata": { ... },
  "canvas": { ... },
  "templates": [ ... ],
  "corridors": [ ... ],
  "nodes": [ ... ],
  "paths": [ ... ],
  "zones": [ ... ],
  "groups": [ ... ],
  "display": { ... },
  "validation": { ... }
}
\end{lstlisting}

\section{Node Object}

\begin{lstlisting}[language=json,caption={Node JSON structure}]
{
  "id": "mach_001",
  "type": "machine",
  "name": "CNC Mill 1",
  "label": "M1",
  "visual": {
    "x": 350,
    "y": 250,
    "width": 80,
    "height": 60,
    "rotation": 0,
    "icon": "cnc_mill",
    "color": "#4A90D9",
    "labelPosition": "bottom",
    "labelVisible": true
  },
  "simulation": {
    "servers": 1,
    "capacity": 1,
    "processTime": {
      "distribution": "triangular",
      "min": 8.0,
      "mode": 10.0,
      "max": 15.0,
      "unit": "minutes"
    },
    "mtbf": 480,
    "mttr": 30
  }
}
\end{lstlisting}

\section{Path Object}

\begin{lstlisting}[language=json,caption={Path JSON structure}]
{
  "id": "path_001",
  "from": "src_001",
  "to": "buf_001",
  "pathType": "single",
  "routingMode": "direct",
  "visual": {
    "color": "#888888",
    "thickness": 2,
    "style": "solid",
    "arrowSize": 8,
    "laneSpacing": 6,
    "waypoints": []
  },
  "simulation": {
    "transportType": "conveyor",
    "speed": 1.0,
    "capacity": 10,
    "lanes": 1,
    "bidirectional": false
  }
}
\end{lstlisting}

% ============================================================================
\chapter{Icon Reference}
% ============================================================================

\section{Available Icons}

\begin{longtable}{llp{6cm}}
\toprule
\textbf{Category} & \textbf{Icon Key} & \textbf{Description} \\
\midrule
\endhead
\multirow{3}{*}{Sources} & source\_arrow & Arrow indicating entry \\
 & source\_funnel & Funnel shape \\
 & source\_truck & Truck/delivery icon \\
\midrule
\multirow{3}{*}{Sinks} & exit\_flag & Finish flag \\
 & exit\_door & Door/exit \\
 & exit\_arrow & Arrow indicating exit \\
\midrule
\multirow{6}{*}{Machines} & machine\_generic & Generic machine \\
 & cnc\_mill & CNC milling machine \\
 & cnc\_lathe & CNC lathe \\
 & robot\_arm & Robotic arm \\
 & press & Press machine \\
 & drill & Drill press \\
\midrule
\multirow{4}{*}{Buffers} & buffer\_fifo & FIFO queue \\
 & buffer\_lifo & LIFO stack \\
 & buffer\_rack & Storage rack \\
 & buffer\_pallet & Pallet storage \\
\midrule
\multirow{3}{*}{Stations} & workstation & General workstation \\
 & assembly & Assembly station \\
 & inspection & Inspection point \\
\midrule
\multirow{3}{*}{Transport} & conveyor & Conveyor belt \\
 & agv & Automated guided vehicle \\
 & junction & Path junction \\
\bottomrule
\caption{Available icons by category}
\end{longtable}

% ============================================================================
% Back matter
% ============================================================================

\chapter*{Document History}
\addcontentsline{toc}{chapter}{Document History}

\begin{table}[h]
\centering
\begin{tabular}{llp{8cm}}
\toprule
\textbf{Version} & \textbf{Date} & \textbf{Changes} \\
\midrule
2.0 & January 2025 & Initial release of v2.0 user manual \\
\bottomrule
\end{tabular}
\end{table}

\end{document}
